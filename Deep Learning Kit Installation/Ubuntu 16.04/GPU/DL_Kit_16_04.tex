\documentclass[12pt]{article}
%\usepackage[pass,letterpaper]{geometry}              %This package needs to be enaled when dvi--->ps----->PDF
\usepackage{stmaryrd, amsfonts, mathptmx, mathtools, array, siunitx, subfigure, graphicx, listings, etoolbox, mdwlist, xcolor }
\usepackage{multirow, indentfirst, cite, verbatim, keyval, textcomp, enumerate, calc, microtype, color,marvosym, enumitem}
\usepackage[export]{adjustbox}
\usepackage[obeyspaces]{url}
\usepackage[T1]{fontenc}
\usepackage[english]{babel}

\lstdefinestyle{DOS}
{
    backgroundcolor=\color{black},
    basicstyle=\scriptsize\color{white}\ttfamily
}



\usepackage[colorlinks,linktocpage,dvips]{hyperref}
\hypersetup{
   colorlinks   = true,                               %Colours links instead of ugly boxes
   urlcolor     = blue,                               %Colour for external hyper links
   linkcolor    = blue,                               %Colour of internal links
   citecolor    = red,                                %Colour of citations
   setpagesize  = false,
   linktocpage  = true,
}

\makeatletter
\patchcmd{\l@section}
  {\hfil}
  {\leaders\hbox{\normalfont$\m@th\mkern \@dotsep mu\hbox{.}\mkern \@dotsep mu$}\hfill}
  {}{}
\makeatother
\definecolor{anti-flashwhite}{rgb}{0.95, 0.95, 0.96}
\lstset{
backgroundcolor=\color{anti-flashwhite},
basicstyle=\small\ttfamily,
columns=flexible,
breaklines=true
}


\newcommand{\shellcmd}[1]{\\\indent\indent\texttt{\footnotesize\$ #1}\\}
\parskip 0.5em
\parindent 2em
\setlength{\textwidth}{6.0in} \setlength{\textheight}{8.8in}
\setlength{\topmargin}{0.4in} \setlength{\headheight}{0.0in}
\setlength{\headsep}{0.0in} \setlength{\oddsidemargin}{0.25in}
\setlength{\parindent}{.2in}
%\renewcommand{\baselinestretch}{1.5}
\setlength{\parindent}{2em}
\setlength{\parskip}{1em}
%%%%%%%%%%%%%%%%%%%%%%%%%%%%%%%%%%%%%%%%%%%%%%%%%%%%%%%%%%%%%%%%%%%%%%%%%%%
\title{\textbf{Google Cloud Deep Learning Kit\\ Ubuntu 16.04 \\ Software Installation\\ GPU }}%

\author{Data Science Program, GWU, USA \\
School of Electrical and Computer Engineering, OSU, USA\\
\vspace{1cm}\\
\Letter : martin.t.hagan@okstate.edu\\
\Letter : ajafari@gwu.edu\\
\Letter : amir.h.jafari@okstate.edu  }
%%%%%%%%%%%%%%%%%%%%%%%%%%%%%%%%%%%%%%%%%%%%%%%%%%%%%%%%%%%%%%%%%%%%%%%%%%%
\begin{document}
\begin{figure}
\centering \includegraphics[width=2in, height=1in]{fig/GW_logo.eps}\hfill
\centering \includegraphics[width=1in, height=1in]{fig/logo1.eps}\hfill
\end{figure}

%%%%%%%%%%%%%%%%%%%%%%%%%%%%%%%%%%%%%%%%%%%%%%%%%%%%%%%%%%%%%%%%%%%%%%%%%%%
\maketitle
\newpage
\tableofcontents
\newpage
%\listoffigures
%\newpage
%\listoftables
% \newpage
%\lstlistoflistings
% \newpage
%%%%%%%%%%%%%%%%%%%%%%%%%%%%%%%%%%%%%%%%%%%%%%%%%%%%%%%%%%%%%%%%%%%%%%%%%%%

\section{Manual installation}
\subsection{Chromium}

\begin{enumerate}
  \item Log in to your google cloud console.
  \item Choose compute engine and create an instance.
  \item Customize the VM and get an instance with 1 GPU.
  \item Set up your SSH key and then create instance.
  \item After the instance is created you can log in couple of ways, SSH browser, Terminal or MobaXterm.
  \item In this tutorial we used MobaXterm.
  \item Now you setup your MobaXterm and use the external IP address as host and your username that you set up for public key.
  \item Log in to your Linux terminal and lets install chromium. Click on the hyperlink below and enter all the commands in the not Notepad one by one.
\end{enumerate}


\begin{center}
\href{run:./Text_Files_16/Chrome.txt}{\Large Chromium}
\end{center}

\subsection{CUDA 8.0 and CUDA-Toolkit}
\begin{enumerate}[resume]
  \item Open your chromium browser in VM by chromium in terminal.
  \item Lets log in to Nvidia developer  \href{https://developer.nvidia.com/cuda-downloads}{website}
  \item Choose Linux, x86\_64, Ubuntu, 16.04 and click on debfile.
  \item Download the deb file and its patch.
  \item Click on the hyperlink below and enter all the commands in the not Notepad one by one.
\end{enumerate}

\begin{center}
\href{run:./Text_Files_16/Cuda.txt}{\Large CUDA}
\end{center}

\begin{enumerate}[resume]
  \item Once nano is open edit the PATH variable to include /usr/local/cuda-8.0/bin folder.
  \item It should be like this:
  \item  {\scriptsize PATH="/usr/local/sbin:/usr/local/bin:/usr/sbin:/usr/bin:/sbin:/bin:/usr/games:/usr/local/games:/usr/local/cuda-8.0/bin"}
  \item After editing this line press Ctrl + X to exit the editor and press Y when prompted whether you want to save it.
  \item Click on the hyperlink below and enter all the commands in the not Notepad one by one.
\end{enumerate}

\begin{center}
\href{run:./Text_Files_16/Cuda1.txt}{\Large CUDA-Continue}
\end{center}


%\subsection{CUDA Toolkit}
%\begin{enumerate}[resume]
%  \item After Successful installation of CUDA 8.0 we need to install nvidia toolkit.
%  \item Click on the hyperlink below and enter all the commands in the not Notepad one by one.
%\end{enumerate}
%
%
%\begin{center}
%\href{run:./Text_Files_16/Cuda_toolkit.txt}{\Large Cuda Toolkit}
%\end{center}

\subsection{cuDNN 6.0}

\begin{enumerate}[resume]
  \item Now, we need to install cuDNN to complete the GPU installation build for frameworks.
  \item Open your chromium browser in VM by typing chromium in terminal.
  \item Lets log in to Nvidia developer  \href{https://developer.nvidia.com/rdp/cudnn-download}{website}.
  \item Next, download the cuDNN 6.0 for Cuda 8.0 for ubuntu 14.04 (cuDNN v6.0 Library for Linux).
  \item Click on the hyperlink below and enter all the commands in the not Notepad one by one.
\end{enumerate}

\begin{center}
\href{run:./Text_Files_16/cuDNN.txt}{\Large cuDNN}
\end{center}

\subsection{Phyton 3.5}

\begin{enumerate}[resume]
  \item Install Python 3
\end{enumerate}

\begin{center}
\href{run:./Text_Files_16/Python3.txt}{\Large Python3}
\end{center}

\subsection{Phyton 2.7}

\begin{enumerate}[resume]
  \item Install Python 2.7
\end{enumerate}

\begin{center}
\href{run:./Text_Files_16/Python2.txt}{\Large Python2.7}
\end{center}

\subsection{Additional Python Packages}

\begin{enumerate}[resume]
  \item Install Additional python packages.
\end{enumerate}

\begin{center}
\href{run:./Text_Files_16/AdPP.txt}{\Large Additional python packages}
\end{center}

\subsection{Pycharm}

\begin{enumerate}[resume]
  \item Open your chromium browser in VM by typing chromium in terminal.
  \item Lets log   \href{http://ppa.launchpad.net/mystic-mirage/pycharm/ubuntu/pool/main/p/pycharm/}{website}.
  \item Download pycharm-community\_2016.3\~mm1\_all.deb file.	
  \item Click on the hyperlink below and enter all the commands in the not Notepad one by one.
\end{enumerate}

\begin{center}
\href{run:./Text_Files_16/Pycharm.txt}{\Large Pycharm}
\end{center}

\begin{enumerate}[resume]
  \item You can run the editor by entering pycharm-community.
\end{enumerate}

\subsection{Deep Learning Frameworks}

\subsubsection{Torch and ZeroBraneStudio}

\begin{enumerate}[resume]
  \item Lets install torch.
\end{enumerate}

\begin{center}
\href{run:./Text_Files_16/Torch.txt}{\Large Torch}
\end{center}

\begin{enumerate}[resume]
  \item Open your chromium browser in VM by typing chromium in terminal.
  \item Lets log in to ZeroBraneStudio \href{https://studio.zerobrane.com/download?not-this-time}{website}.
  \item Download the Linux version.	
  \item Click on the hyperlink below and enter all the commands in the not Notepad one by one.
\end{enumerate}

\begin{center}
\href{run:./Text_Files_16/ZeroB.txt}{\Large ZeroBraneStudio}
\end{center}

\begin{enumerate}[resume]
  \item You can run the editor by entering zbstudio.
  \item To test that the Torch is working. Download gputest.luad from \href{https://gist.github.com/jaderberg/6436387}{here}.
  \item Click on the hyperlink below and enter all the commands in the not Notepad one by one.
\end{enumerate}

\begin{center}
\href{run:./Text_Files_16/Torch_test.txt}{\Large Torch Test}
\end{center}


\subsubsection{Caffe}

\begin{enumerate}[resume]
  \item Lets install Caffe. Run the following command in the hyperlink below
  \item Disregard the next 2 steps if you did not get any error.
  \item You could have "TypeError: 'NoneType' object is not callable" error when installing pillow, then try:
  \item sudo apt-get install pypy-dev
\end{enumerate}

\begin{center}
\href{run:./Text_Files_16/Caffe.txt}{\Large Caffe}
\end{center}

\begin{enumerate}[resume]
  \item Edit the following command after the vi Makefile.config.
  \item Uncomment the line: USE\_CUDNN := 1
  \item Make sure the CUDA\_DIR correctly points to our CUDA installation.
  \item If you want the Matlab wrapper, uncomment the appropriate MATLAB\_DIR line.
  \item Now we build Caffe. Set X to the number of CPU threads (or cores) on your machine (in the Notepad jX). Use the command htop to check how many CPU threads you have.
  \item Click on the hyperlink below and enter all the commands in the not Notepad one by one.
  PYTHON_INCLUDE := /usr/include/python2.7 \
/usr/local/lib/python2.7/dist-packages/numpy/core/include/ \
                /usr/lib/python2.7/dist-packages/numpy/core/include
                
                WITH_PYTHON_LAYER := 1
                
                INCLUDE_DIRS := $(PYTHON_INCLUDE) /usr/local/include /usr/include/hdf5/serial
LIBRARY_DIRS := $(PYTHON_LIB) /usr/local/lib /usr/lib
CUDA_DIR := /usr/local/cuda-8.0

\end{enumerate}

\begin{center}
\href{run:./Text_Files_16/Caffe1.txt}{\Large Caffe Continue 1}
\end{center}

\begin{enumerate}[resume]
   \item After you did the sudo nano \~/.bashrc. At the end of the script add the following command. Ctrl+x and hit the y key on your keyboard and enter.
   \item export PYTHONPATH=/home/ajafari/caffe/python
   \item Now run the all the commands in the Notepad and Caffe should point to directory to python path.
\end{enumerate}

\begin{center}
\href{run:./Text_Files_16/Caffe2.txt}{\Large Caffe Continue 2}
\end{center}

\begin{enumerate}[resume]
   \item Open python and import Caffe to test everything is working.
\end{enumerate}


\subsubsection{Tensorflow}

\begin{enumerate}[resume]
  \item Lets install tensoflow gpu.
\end{enumerate}

\begin{center}
\href{run:./Text_Files_16/tensorflow.txt}{\Large Tensorflow GPU}
\end{center}

\begin{enumerate}[resume]
  \item To activate the tensorflow for python 2.7 enter source \~/tensorflow2/bin/activate.
  \item To activate the tensorflow for python 3.5 enter source \~/tensorflow3/bin/activate.
  \item Then you can open up pycharm and choose the interpreter based on the python that you choosed.
\end{enumerate}

\subsubsection{Theano}

\begin{enumerate}[resume]
  \item Lets install theano.
\end{enumerate}

\begin{center}
\href{run:./Text_Files_16/Theano.txt}{\Large Theano}
\end{center}

\begin{enumerate}[resume]
  \item To check theano work on gpu, you need to set the flags.
  \item Follow the instruction on the website \href{http://theano.readthedocs.io/en/0.8.x/tutorial/using_gpu.html}{website}
\end{enumerate}

\subsubsection{Keras}

\begin{enumerate}[resume]
  \item Lets install keras.
\end{enumerate}

\begin{center}
\href{run:./Text_Files_16/Keras.txt}{\Large Keras}
\end{center}

\subsubsection{Pytorch}

\begin{enumerate}[resume]
  \item Lets install Pytorch.
\end{enumerate}

\begin{center}
\href{run:./Text_Files_16/Pytorch.txt}{\Large Pytorch}
\end{center}





%\newpage
%\addcontentsline{toc}{section}{References}
%\bibliographystyle{IEEEtran}
%\bibliography{mybibInstallation}
\end{document}
