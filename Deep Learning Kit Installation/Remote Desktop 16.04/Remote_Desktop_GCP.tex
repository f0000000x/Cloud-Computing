\documentclass[12pt]{article}
%\usepackage[pass,letterpaper]{geometry}              %This package needs to be enaled when dvi--->ps----->PDF
\usepackage{stmaryrd, amsfonts, mathptmx, mathtools, array, siunitx, subfigure, graphicx, listings, etoolbox, mdwlist, xcolor }
\usepackage{multirow, indentfirst, cite, verbatim, keyval, textcomp, enumerate, calc, microtype, color,marvosym, enumitem}
\usepackage[export]{adjustbox}
\usepackage[obeyspaces]{url}
\usepackage[T1]{fontenc}
\usepackage[english]{babel}

\lstdefinestyle{DOS}
{
    backgroundcolor=\color{black},
    basicstyle=\scriptsize\color{white}\ttfamily
}



\usepackage[colorlinks,linktocpage,dvips]{hyperref}
\hypersetup{
   colorlinks   = true,                               %Colours links instead of ugly boxes
   urlcolor     = blue,                               %Colour for external hyper links
   linkcolor    = blue,                               %Colour of internal links
   citecolor    = red,                                %Colour of citations
   setpagesize  = false,
   linktocpage  = true,
}

\makeatletter
\patchcmd{\l@section}
  {\hfil}
  {\leaders\hbox{\normalfont$\m@th\mkern \@dotsep mu\hbox{.}\mkern \@dotsep mu$}\hfill}
  {}{}
\makeatother
\definecolor{anti-flashwhite}{rgb}{0.95, 0.95, 0.96}
\lstset{
backgroundcolor=\color{anti-flashwhite},
basicstyle=\small\ttfamily,
columns=flexible,
breaklines=true
}


\newcommand{\shellcmd}[1]{\\\indent\indent\texttt{\footnotesize\$ #1}\\}
\parskip 0.5em
\parindent 2em
\setlength{\textwidth}{6.0in} \setlength{\textheight}{8.8in}
\setlength{\topmargin}{0.4in} \setlength{\headheight}{0.0in}
\setlength{\headsep}{0.0in} \setlength{\oddsidemargin}{0.25in}
\setlength{\parindent}{.2in}
%\renewcommand{\baselinestretch}{1.5}
\setlength{\parindent}{2em}
\setlength{\parskip}{1em}
%%%%%%%%%%%%%%%%%%%%%%%%%%%%%%%%%%%%%%%%%%%%%%%%%%%%%%%%%%%%%%%%%%%%%%%%%%%
\title{\textbf{Remote Desktop\\Google Cloud \\ Ubuntu 16.04 }}%

\author{Data Science Program, GWU, USA \\
School of Electrical and Computer Engineering, OSU, USA\\
\vspace{1cm}\\
\Letter : martin.t.hagan@okstate.edu\\
\Letter : ajafari@gwu.edu\\
\Letter : amir.h.jafari@okstate.edu  }
%%%%%%%%%%%%%%%%%%%%%%%%%%%%%%%%%%%%%%%%%%%%%%%%%%%%%%%%%%%%%%%%%%%%%%%%%%%
\begin{document}
\begin{figure}
\centering \includegraphics[width=2in, height=1in]{fig/GW_logo.eps}\hfill
\centering \includegraphics[width=1in, height=1in]{fig/logo1.eps}\hfill
\end{figure}
%%%%%%%%%%%%%%%%%%%%%%%%%%%%%%%%%%%%%%%%%%%%%%%%%%%%%%%%%%%%%%%%%%%%%%%%%%%
\maketitle
\newpage
\tableofcontents
\newpage
%\listoffigures
%\newpage
%\listoftables
% \newpage
%\lstlistoflistings
% \newpage
%%%%%%%%%%%%%%%%%%%%%%%%%%%%%%%%%%%%%%%%%%%%%%%%%%%%%%%%%%%%%%%%%%%%%%%%%%%

\section{Remote Desktop}

\begin{enumerate}
  \item Go to \href{https://console.cloud.google.com} {Google Cloud Console}(sign in with your login credentials)
  \item Lets connect to your VM.
  \item Now you should have your Linux terminal.
  \item Click on the hyper link below and enter all the commands. By entering these commands you are changing the firewall rule and install unity desktop (Note: if a screen pops up select select lightdm  then ok.).
\end{enumerate}


\begin{center}
\href{run:./Text_Files_14/First.txt}{\Large Remote Desktop}
\end{center}

\begin{enumerate}[resume]
  \item After entering vncserver, it will prompt you for a password less than 8 characters
  \item Make your own password.
  \item Say no when asked if you want to make a read-only password.
  \item Now you should be able to get the log file as follows, remember the log number.
  \item Log file is /home/mike/.vnc/instance-1:1.log
  \item Click on the hyper link below and enter all the commands.
\end{enumerate}

\begin{center}
\href{run:./Text_Files_14/Second.txt}{\Large Remote Desktop Continue}
\end{center}

\begin{enumerate}[resume]
  \item Now we need to edit the newly-created startup file.
  \item Use vim ,nano, or any editor (nano \~/.vnc/xstartup) you are familiar with and paste the following line in the link belew into your editor.
\end{enumerate}

\begin{center}
\href{run:./Text_Files_14/Third.txt}{\Large Remote Desktop Continue}
\end{center}

\begin{enumerate}[resume]
  \item Once you have finished editing the script exit out.
  \item Click on the hyper link below and enter all the commands.
\end{enumerate}

\begin{center}
\href{run:./Text_Files_14/Forth.txt}{\Large Remote Desktop Continue}
\end{center}

\begin{enumerate}[resume]
  \item Now remember the log file number.
  \item Use the external IP address that you had in your VM from dashboard and open up the vncviewer or safari in Mac,
  \item Finally enter the IP address followed with the colon and the number.
\end{enumerate}











%\newpage
%\addcontentsline{toc}{section}{References}
%\bibliographystyle{IEEEtran}
%\bibliography{mybibInstallation}
\end{document}
